\documentclass[]{article}
\usepackage[round]{natbib}

\usepackage{fullpage}
\usepackage{url}
\usepackage{authblk}
\usepackage{graphicx}
\usepackage{color}

% need to have the pythonhighlight.sty in directory
\usepackage{pythonhighlight}

% local definitions
\newcommand{\aprcomment}[1]{{\textcolor{blue}{APR: #1}}}

\newcommand{\moments}{\texttt{moments}\xspace}

\begin{document}

\title{Demographic inference using the SFS with Moments and Demes}
\author[1]{Nick Collier}
%\author[2]{Simon Gravel(?)}
\author[1,*]{Aaron P. Ragsdale}
\affil[1]{Department of Integrative Biology, University of Wisconsin--Madison}
%\affil[2]{McGill University}
\affil[*]{apragsdale@wisc.edu}
\maketitle

\begin{abstract}
    Placeholder
\end{abstract}

\section{Introduction}

The genetic composition within a sample of individuals is shaped by their
genome biology and evolutionary history. Variation resulting from this history
is fully described by the ancestral relationships among samples at each locus
in the genome and how they change along a chromosome due to recombination (that
is, information stored in the Ancestral Recombination Graph [cites]). However,
the ARG can be large and unwieldy, and methods for reconstructing history
directly from the ARG, while showing promise [cite], are in their infancy and
so far limited in application and scalability [cite]. Instead, evolutionary
inference using informative summaries provides a tractable and powerful
alternative for learning parameters of population history, natural selection and
genome biology.

One such summary that has seen wide use is the site frequency spectrum (SFS),
which stores the counts (observed or expected) of alleles carried by a given
number of genomes in a set of samples. Like any summary of the data, the SFS
discards information relative to the ARG -- in this case, loci are treated
independently so that haplotypic information is lost. Nonetheless, a great deal
can be learned from the 

What moments is, what type of data it uses, general framework for inference
\begin{enumerate}
    \item The SFS -- how its shape is sensitive to population,
        selection and genome biological parameters
    \item Demographic inference from the SFS
    \item Extensions: inference of selection, mutation rates, etc
    \item Extensions: Multi-allele and multi-locus models
\end{enumerate}

\section{Interface with Demes}

\begin{enumerate}
    \item Manipulating and plotting Demes-specified demographic models
    \item Computing statistics given a Demes model
    \item Inferring history on a demes model, including specifying parameters to be fit
\end{enumerate}

\section{Examples}

Two examples, one using simulated data with known parameters, which we try to
reinfer. And another with actual data (humans + neanderthal). We briefly
describe these, highlighting the high-level components of the fits. For
detailed information for each, including managing data, specifying models,
performing inference, and visualizing results, we refer readers to the GitHub
(\url{}), which is maintained with up-to-date versioning.

\subsection{Inferring parameters in a simulated isolation-with-migration model}

A python snippet:
\begin{python}
dbg = msprime.DemographyDebugger(
  population_configurations=population_configurations,
  migration_matrix=migration_matrix,
  demographic_events=demographic_events)
dbg.print_history()
ts = msprime.simulate(
  population_configurations=population_configurations,
  migration_matrix=migration_matrix,
  demographic_events=demographic_events)
\end{python}
or
\begin{python}
dbg = demography.debug()
dbg.print_history()
ts = msprime.simulate(demography=demography)
\end{python}

This was in discussion of \citet{kelleher2016efficient}.

\subsection{Inferring human-Neanderthal demographic parameters}

\begin{python}
def myFunc():
    pass
\end{python}

\section{Considerations and caveats}

\begin{enumerate}
    \item Incluce here general ideas about the strengths and weaknesses of
        various approaches in population genetic inference, including when
        using the SFS.
    \item Challenges in finding local/global optima.
    \item Challenges in exploring parameter space.
    \item Gene dense vs gene sparse genomic architectures among species.
\end{enumerate}

\bibliographystyle{plainnat}
\bibliography{paper}

\clearpage

\appendix
\renewcommand{\thesection}{A\arabic{section}}
\renewcommand{\theequation}{A\arabic{equation}}
\renewcommand{\thefigure}{A\arabic{figure}}
\renewcommand{\thetable}{A\arabic{table}}
\setcounter{figure}{0}
\setcounter{equation}{0}
\setcounter{table}{0}

%\section{Appendix} \label{appendix}
\end{document}
